\begin{lstlisting}
/**
* Funcion que posiciona las defensas y que tiene implicita la funcion objetivo del algoritmo Voraz
* @param freeCells matriz de numero de celdas-ancho por numero de celdas-alto, contiene true si el centro de la celda
*                  esta libre y false si el centro de la celda esta ocupado por un obstaculo.
* @param nCellsWidth numero de celdas en anchura.
* @param nCellsHeight numero de celdas en altura.
* @param mapWidth ancho total del mapa preestablecido.
* @param mapHeight alto total del mapa preestablecido.
* @param obstacles Lista actual de Obstaculos creados y situados.
* @param defenses Lista actual de Defensas creadas, no situadas.
*/
void DEF_LIB_EXPORTED placeDefenses(bool** freeCells, int nCellsWidth, int nCellsHeight, float mapWidth, float mapHeight
, std::list<Object*> obstacles, std::list<Defense*> defenses)
{
	float cellWidth = mapWidth / nCellsWidth;
	float cellHeight = mapHeight / nCellsHeight;
	float **cellsValues = new float*[nCellsWidth];
	/**
	* Matriz de valores del algoritmo voraz.
	*/
	cellsValues = matrixValues(freeCells, nCellsWidth, nCellsHeight, mapWidth, mapHeight, obstacles, defenses);
	/**
	* Lista para almacenar la matriz de valores y ordenarlos
	*/
	std::list<ValueList> VectorValues;
	Vector3 v3tmp{}, cellSelection{};
	int maxAttemps = 1000, n = 0;
	
	for (int i = 0; i < nCellsWidth; ++i)
	{
		for (int j = 0; j < nCellsHeight; ++j)
		{
			v3tmp.x = (float)i;
			v3tmp.y = (float)j;
			VectorValues.emplace_back(cellsValues[i][j], v3tmp);
		}
	}
	
	/**
	* Ordenacion de los valores de la lista.
	*/
	VectorValues.sort(std::greater<ValueList>());
	
	std::list<ValueList>::iterator currentValue = VectorValues.begin();
	
	List<Defense*>::iterator currentDefense = defenses.begin();
	
	while(currentDefense != (defenses.end()) and maxAttemps > 0)
	{
		/**
		* funcion select, primera defensa en el centro
		*/
		if(currentDefense == defenses.begin())
			cellSelection = cellSelect(currentValue, freeCells, nCellsWidth, nCellsHeight, 1);
		else
		{
			cellSelection = cellSelect(currentValue, freeCells, nCellsWidth, nCellsHeight);
		}
		
		(*currentDefense)->position.x = (cellSelection.x * cellWidth) + (cellWidth * 0.5f); //(int)(_RAND2(nCellsWidth))* cellWidth) + cellWidth * 0.5f
		(*currentDefense)->position.y = (cellSelection.y * cellHeight) + (cellHeight * 0.5f); //(int)(_RAND2(nCellsHeight)) *cellHeight) + cellHeight *0.5f
		(*currentDefense)->position.z = 0;// cellSelection.z;
		if(factibility(*currentDefense, defenses, obstacles, mapWidth, mapHeight))
		{
			(*currentDefense)->health = DEFAULT_DEFENSE_HEALTH;
			++currentDefense;
			++currentValue;
			//quitar de la lista
			VectorValues.pop_front();
		}
		else
		{
			++currentValue;
			//quitar de la lista
			VectorValues.pop_front();
		}
	}
	
	#ifdef PRINT_DEFENSE_STRATEGY
	
	float** cellValues = new float* [nCellsHeight];
	for(int i = 0; i < nCellsHeight; ++i)
	{
		cellValues[i] = new float[nCellsWidth];
		for(int j = 0; j < nCellsWidth; ++j)
		{
			cellValues[i][j] = (int)(cellsValues[i][j]) % 256;
		}
	}
	dPrintMap("defenseValueCellsHead.ppm", nCellsHeight, nCellsWidth, cellHeight, cellWidth, freeCells
	, cellValues, std::list<Defense*>(), true);
	
	for(int i = 0; i < nCellsHeight ; ++i)
	delete [] cellValues[i];
	delete [] cellValues;
	cellValues = nullptr;
	
	#endif
	for(int i = 0; i < nCellsHeight ; ++i)
	delete [] cellsValues[i];
	delete [] cellsValues;
	cellsValues = nullptr;
}
\end{lstlisting}
Mi estrategia está formada por un algoritmo voraz que selecciona las celdas más valiosas en primer lugar mediante una lista rellenada por una matriz de puntuaciones, apoyándose en la función de factibilidad para descartar las posiciones del mapa ya ocupadas y en segundo lugar lo más cercano al centro del mapa mas varias en los bordes como protección de avanzadilla.