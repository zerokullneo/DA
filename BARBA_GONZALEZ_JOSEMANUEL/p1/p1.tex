\documentclass[]{article}

\usepackage[left=2.00cm, right=2.00cm, top=2.00cm, bottom=2.00cm]{geometry}
\usepackage[spanish,es-noshorthands]{babel}
\usepackage[utf8]{inputenc} % para tildes y ñ
\usepackage{graphicx} % para las figuras
\usepackage{xcolor}
\usepackage{listings} % para el código fuente en c++

\lstdefinestyle{customc}{
  belowcaptionskip=1\baselineskip,
  breaklines=true,
  frame=single,
  xleftmargin=\parindent,
  language=C++,
  showstringspaces=false,
  basicstyle=\footnotesize\ttfamily,
  keywordstyle=\bfseries\color{green!40!black},
  commentstyle=\itshape\color{gray!40!gray},
  identifierstyle=\color{black},
  stringstyle=\color{orange},
}
\lstset{style=customc}


%opening
\title{Práctica 1. Algoritmos devoradores}
\author{\input{José Manuel Barba González}}


\begin{document}

%\maketitle

%\begin{abstract}
%\end{abstract}

% Ejemplo de ecuación a trozos
%
%$f(i,j)=\left\{ 
%  \begin{array}{lcr}
%      i + j & si & i < j \\ % caso 1
%      i + 7 & si & i = 1 \\ % caso 2
%      2 & si & i \geq j     % caso 3
%  \end{array}
%\right.$

\begin{enumerate}
\item Describa a continuación la función diseñada para otorgar un determinado valor a cada una de las celdas del terreno de batalla para el caso del centro de extracción de minerales. 

$$ f(radio,rango,salud)=(radio+rango)-1+rango^2+3\frac{radio}{salud} $$

Escriba aquí su respuesta al ejercicio 1.

\item Diseñe una función de factibilidad explicita y descríbala a continuación.

\begin{lstlisting}
	/**
	* Funcion que comprueba la factibilidad de posicionar una defensa en una posicion dada del mapa.
	* @param currentDefense Defensa actual a evaluar su posicion.
	* @param defenses Lista actual de Defensas creadas y situadas.
	* @param obstacles Lista actual de Obstaculos creados y situados.
	* @param mapWidth ancho total del mapa preestablecido.
	* @param mapHeight alto total del mapa preestablecido.
	* @return Devuelve true si se puede colocar y false si no se puede colocar.
	*/
	bool factibility(Defense* currentDefense, std::list<Defense*> defenses, std::list<Object*> obstacles, float mapWidth, float mapHeight)
\end{lstlisting}

Comprueba que no se salga de los bordes y que no choque contra un obstáculo u defensa.

\item A partir de las funciones definidas en los ejercicios anteriores diseñe un algoritmo voraz que resuelva el problema para el caso del centro de extracción de minerales. Incluya a continuación el código fuente relevante. 

\begin{lstlisting}
// sustituya este codigo por su respuesta
void placeDefenses(...) {

    List<Defense*>::iterator currentDefense = defenses.begin();
    while(currentDefense != defenses.end() && maxAttemps > 0) {

        (*currentDefense)->position.x = ((int)(_RAND2(nCellsWidth))) * cellWidth + cellWidth * 0.5f;
        ...
        ++currentDefense;
    }
}
\end{lstlisting}

\item Comente las características que lo identifican como perteneciente al esquema de los algoritmos voraces. 

\begin{lstlisting}
	/**
	* Funcion que selecciona la mejor posicion dado los valores de la matriz cellsvalues.
	* @param valuelist iterador que contiene el valor actual de una posicion dada del mapa.
	* @param freeCells matriz de numero de celdas-ancho por numero de celdas-alto, contiene true si el centro de la celda
	*                  esta libre y false si el centro de la celda esta ocupado por un obstaculo.
	* @param nCellsWidth numero de celdas en anchura.
	* @param nCellsHeight numero de celdas en altura.
	* @param extraction Bit que indica si es o no el centro de extraccion.
	* @return Devuelve un tipo Vector3 con la posicion mas prometedora para la defensa.
	*/
	Vector3 cellSelect(std::list<ValueList>::iterator valuelist, bool** freeCells, int nCellsWidth, int nCellsHeight, int extraction = 0)
\end{lstlisting}
En mi caso, la función ``cellSelect'' se comporta eligiendo paulatinamente la primera posición más valiosa de la matriz de valores, sin tener en cuenta la posición que ocupa en el mapa dicho valor, y comportándose de forma voraz.

\item Describa a continuación la función diseñada para otorgar un determinado valor a cada una de las celdas del terreno de batalla para el caso del resto de defensas. Suponga que el valor otorgado a una celda no puede verse afectado por la colocación de una de estas defensas en el campo de batalla. Dicho de otra forma, no es posible modificar el valor otorgado a una celda una vez que se haya colocado una de estas defensas. Evidentemente, el valor de una celda sí que puede verse afectado por la ubicación del centro de extracción de minerales.

Escriba aquí su respuesta al ejercicio 5.

\item A partir de las funciones definidas en los ejercicios anteriores diseñe un algoritmo voraz que resuelva el problema global. Este algoritmo puede estar formado por uno o dos algoritmos voraces independientes, ejecutados uno a continuación del otro. Incluya a continuación el código fuente relevante que no haya incluido ya como respuesta al ejercicio 3. 

\begin{lstlisting}
// sustituya este codigo por su respuesta
void placeDefenses(...) {

    List<Defense*>::iterator currentDefense = defenses.begin();
    while(currentDefense != defenses.end() && maxAttemps > 0) {

        (*currentDefense)->position.x = ((int)(_RAND2(nCellsWidth))) * cellWidth + cellWidth * 0.5f;
        ...
        ++currentDefense;
    }
}
\end{lstlisting}

\end{enumerate}

Todo el material incluido en esta memoria y en los ficheros asociados es de mi autoría o ha sido facilitado por los profesores de la asignatura. Haciendo entrega de este documento confirmo que he leído la normativa de la asignatura, incluido el punto que respecta al uso de material no original.

\end{document}
