\begin{lstlisting}
/**
* Funcion que posiciona las defensas y que tiene implicita la funcion objetivo del algoritmo Voraz
* @param freeCells matriz de numero de celdas-ancho por numero de celdas-alto, contiene true si el centro de la celda
*                  esta libre y false si el centro de la celda esta ocupado por un obstaculo.
* @param nCellsWidth numero de celdas en anchura.
* @param nCellsHeight numero de celdas en altura.
* @param mapWidth ancho total del mapa preestablecido.
* @param mapHeight alto total del mapa preestablecido.
* @param obstacles Lista actual de Obstaculos creados y situados.
* @param defenses Lista actual de Defensas creadas, no situadas.
*/
void DEF_LIB_EXPORTED placeDefenses(bool** freeCells, int nCellsWidth, int nCellsHeight, float mapWidth, float mapHeight
, std::list<Object*> obstacles, std::list<Defense*> defenses)
\end{lstlisting}

En mi caso, el algoritmo es voraz porque crea una matriz de valores entre 100 y 10 puntos a las casillas, para luego ordenarlas en una lista de mayor a menor valor, y terminar colocando las defensas, primero el centro de extracción en el centro del mapa, como segunda opción colocar ciertas defensas en los bordes del mapa o lo más cercano posible, por último, en casillas cercanas al centro del mapa.