% Elimine los símbolos de tanto por ciento para descomentar las siguientes instrucciones e incluir una imagen en su respuesta. La mejor ubicación de la imagen será determinada por el compilador de Latex. No tiene por qué situarse a continuación en el fichero en formato pdf resultante.
\begin{figure}
\centering
\includegraphics[width=0.7\linewidth]{./defenseValueCellsHead} % no es necesario especificar la extensión del archivo que contiene la imagen
\caption{Estrategia devoradora para la mina}
\label{fig:defenseValueCellsHead}
\end{figure}

\begin{lstlisting}
	/**
	* Dada una celda o posición devuelve un valor para crear en otra función la matriz de valores de las celdas.
	* @param row Fila a evaluar.
	* @param col Columna a evaluar.
	* @param freeCells matriz de numero de celdas-ancho por número de celdas-alto, contiene true si el centro de la celda
	*                  esta libre y false si el centro de la celda esta ocupado por un obstáculo.
	* @param nCellsWidth número de celdas en anchura.
	* @param nCellsHeight número de celdas en altura.
	* @param mapWidth ancho total del mapa preestablecido.
	* @param mapHeight alto total del mapa preestablecido.
	* @param obstacles Lista actual de Obstáculos creados y situados.
	* @param defenses Lista actual de Defensas creadas, no situadas.
	* @return Devuelve un valor determinado para una posición determinada del tablero.
	*/
	float cellValue(int row, int col, placePosition place, bool** freeCells, int nCellsWidth, int nCellsHeight
	, float mapWidth, float mapHeight, List<Object*> obstacles, List<Defense*> defenses)
\end{lstlisting}

Otorga 100 puntos a las posiciones más valiosas, en este caso las centrales, las segundas más valiosas de 90 puntos a los bordes, 20 puntos a las posiciones cercanas al centro y 10 puntos al resto de posiciones.