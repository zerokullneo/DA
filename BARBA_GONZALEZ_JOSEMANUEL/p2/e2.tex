Estructura de la tabla, en la matriz abajo descrita se guarda la tabla de subproblemas resueltos del algoritmo de la mochila.
\begin{lstlisting}
	std::vector< std::vector<float> > defensesAses(tlpWidth, std::vector<float>(tlpHeight));
\end{lstlisting}

Lista de las defensas, se usa la lista de las defensas para realizar el calculo de la valoración dada por la función ``defensiveCapacity''.
\begin{lstlisting}
	std::list<Defense*> defenses
\end{lstlisting}

Valor del peso de cada defensa.
\begin{lstlisting}
	(*itDefenses)->cost
\end{lstlisting}

Valor de la capacidad defensiva de cada defensa, calculada individualmente por la función ``defensiveCapacity''.
\begin{lstlisting}
	float value;
	value = defensiveCapacity(itDefenses);
\end{lstlisting}
