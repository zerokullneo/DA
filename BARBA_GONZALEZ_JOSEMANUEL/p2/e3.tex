\begin{lstlisting}
/**
* Función que dada una defensa devuelva su capacidad o valor defensivo. La capacidad o valor defensiva
* es una valoracion de los atributos health y cost mas attacksPerSecond y damage mas range.
* @param currentDefense Iterador a la defensa actual para calcular su valor.
* @return Devuelve un valor tipo float simulando la capacidad defensiva de la defensa.
*/
float defensiveCapacity(std::list<Defense*>::iterator currentDefense)
{
	return ((*currentDefense)->health / (float)(*currentDefense)->cost)
	+ ((*currentDefense)->attacksPerSecond + (*currentDefense)->damage)
	+ ((*currentDefense)->range / 10);
}

/**
* Funcion  que rellena la tabla de subproblemas resueltos con el algoritmo de la mochila
* @param defenses Lista de defensas disponibles para adquirir.
* @param ases Cantidad de ases/monedas para adquirir las defensas.
* @return Devuelve la tabla de subproblemas resueltos en forma de matriz, con forma de vector de vectores.
*/
std::vector< std::vector<float> > tableResults(std::list<Defense*> defenses, unsigned int ases)
{
	size_t tlpWidth = defenses.size();//objetos
	size_t tlpHeight = ases;//capacidad
	std::vector< std::vector<float> > defensesAses(tlpWidth, std::vector<float>(tlpHeight));
	float value;
	
	std::list<Defense*>::iterator itDefenses = defenses.begin();
	
	/**
	* valor = ((*currentDefense)->health / (float)(*currentDefense)->cost) + ((*currentDefense)->attacksPerSecond + (*currentDefense)->damage)
	* peso = (*itDefenses)->cost
	* salttamos de la lista de defensas la primera que ya ha sido comprada.
	*/
	++itDefenses;
	for (int j = 0; j < tlpHeight; ++j)
	{
		if ( j < (*itDefenses)->cost )
		defensesAses[0][j] = 0;
		else
		defensesAses[0][j] = defensiveCapacity(itDefenses);
	}
	
	for (int i = 1; i < tlpWidth; ++i, ++itDefenses)
	{
		value = defensiveCapacity(itDefenses);
		for (int j = 0; j < tlpHeight; ++j)
		{
			if (j < (*itDefenses)->cost)
			defensesAses[i][j] = defensesAses[i-1][j];
			else
			defensesAses[i][j] = std::max(defensesAses[i-1][j], defensesAses[i-1][j - (*itDefenses)->cost] + value);
		}
	}

	return defensesAses;
}
\end{lstlisting}