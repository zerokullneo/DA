La estructura para englobar los datos a ordenar y para seleccionar la posición de la lista de defensas es la clase ``ValueList'' con los atributos ``value'' y ``position'', dicha clase se usa para construir un vector de la biblioteca standard ``std::vector$<$ValueList$>$'' en cada función de ordenación y ordenar el vector previamente a seleccionar su posición en el mapa.

Se sobrecargan los operadores ``$<$'' ``$>$'' ``$<=$'' ``$>=$'' para las comparaciones en los algoritmos de ordenación.

\begin{lstlisting}
class ValueList
{
	public:
		ValueList(float vl=0, Vector3 ps=0): value(vl), position(ps){};
		float value;
		Vector3 position;
};
bool operator <(const ValueList &A, const ValueList &B){ return A.value < B.value; }
bool operator >(const ValueList &A, const ValueList &B){ return A.value > B.value; }
bool operator <=(const ValueList &A, const ValueList &B){ return A.value <= B.value; }
bool operator >=(const ValueList &A, const ValueList &B){ return A.value >= B.value; }
\end{lstlisting}

A continuación se muestra la función que crea la matriz que da valores a las celdas del mapa mediante la funcion ``defaultCellValue()'', para luego ser introducidas en el vector ``std::vector$<$ValueList$>$'' y ser ordenadas en su caso.
\begin{lstlisting}
/**
* Funcion que crea una matriz de valores correspondientes a cada celda del mapa dados por la funcion cellValue, siendo
* la funcion Candidatos del algoritmo Voraz
* @param freeCells matriz de numero de celdas-ancho por numero de celdas-alto, contiene true si el centro de la celda
*                  esta libre y false si el centro de la celda esta ocupado por un obstaculo.
* @param nCellsWidth numero de celdas en anchura.
* @param nCellsHeight numero de celdas en altura.
* @param mapWidth ancho total del mapa preestablecido.
* @param mapHeight alto total del mapa preestablecido.
* @param obstacles Lista actual de Obstaculos creados y situados.
* @param defenses Lista actual de Defensas creadas, no situadas.
* @return Devuelve una matriz rellena con los valores de las posiciones candidatas prometedoras.
*/
float** matrixValues(bool** freeCells, int nCellsWidth, int nCellsHeight
, float mapWidth, float mapHeight, const List<Object*> &obstacles, const List<Defense*> &defenses)
{
	float **matrixOfValues; matrixOfValues = new float* [nCellsWidth];
	float MAXV = 100.0f;
	float MEDV = MAXV * 0.9f;

	for (int i = 0; i < nCellsWidth; ++i)
		matrixOfValues[i] = new float [nCellsHeight];

	for(int iW = nCellsWidth; iW < nCellsWidth; iW++)
		for(int iH = nCellsHeight; iH < nCellsHeight; iH++)
			matrixOfValues[iW][iH] = defaultCellValue(iW, iH, freeCells, nCellsWidth, nCellsHeight, mapWidth,
			mapHeight, obstacles, defenses);

return matrixOfValues;
}
\end{lstlisting}