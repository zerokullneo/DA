Método de ordenación por Fusión descrito en la teoría por pseudocódigo y transcrita a código C++.

\begin{lstlisting}
/**
 * Funcion secundaria de Ordenacion por Fusion que realiza los intercambios indicodos por los parametros de entrada.
 * @param v Variable que tiene el estado actual del vector pasado por referencia.
 * @param i Variable que recibe el inicio de tramo a evaluar.
 * @param k Variable que recibe el tramo de posiciones a evaluar.
 * @param j Variable que recibe el final de tramo a evaluar.
 */
void Fusion(std::vector<ValueList>& v, size_t i, size_t k, size_t j)
{
	size_t n = j - i + 1;
	size_t p = i; size_t q = k + 1;
	std::vector<ValueList> w;

	for(int it = 0; it < n; ++it)
	{
		if (p <= k and (q > j or v[p].value <= v[q].value))
		{
			w.push_back(v[p]);
			p++;
		}
		else
		{
			w.push_back(v[q]);
			q++;
		}
	}

	for (int it = 0; it < n; ++it)
		v[i + it] = w[it];
}

/**
 * Funcion que ordena por el metodo "Fusion", en este caso, un vector de la clase "ValueList" por su atributo "value".
 * @param orderCells Variable del vector inicial pasado por referencia.
 * @param pos_i Variable con la posicion inicial del vector.
 * @param pos_j Variable con la posicion final del vector.
 */
void orderFusion(std::vector<ValueList>& orderCells, size_t pos_i, size_t pos_j)
{
	size_t n = pos_j - pos_i + 1;
	size_t n0 = 3, pos_k;

	if (n <= n0)
		std::sort(orderCells.begin() + pos_i, orderCells.begin() + pos_j, std::greater<ValueList>());
	else
	{
		pos_k = pos_i - 1 + n / 2;
		orderFusion(orderCells, pos_i, pos_k);
		orderFusion(orderCells, pos_k + 1, pos_j);
		Fusion(orderCells, pos_i, pos_k, pos_j);
	}
}
\end{lstlisting}